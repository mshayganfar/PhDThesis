\documentclass[12pt]{report}

\usepackage[titletoc]{appendix}
\usepackage{datetime}
\usepackage{graphicx}
\usepackage{amsmath}
\usepackage{algorithmicx}
\usepackage{algorithm}
\usepackage[explicit]{titlesec}
\usepackage{tocloft}
\usepackage{amssymb}
\usepackage{longtable}
\usepackage{multirow}
\usepackage[english]{babel}

\usepackage{subcaption}
\usepackage{subfig}

\usepackage{url}

\renewcommand{\familydefault}{\rmdefault}
\renewcommand\cftchapaftersnum{.}
\renewcommand\cftchapdotsep{\cftdotsep}

\setlength{\textheight}{8.63in}
\setlength{\textwidth}{5.9in}
\setlength{\topmargin}{-0.2in}
\setlength{\oddsidemargin}{0.3in}
\setlength{\evensidemargin}{0.3in}
\setlength{\headsep}{0.0in}

\titleformat{\chapter}[block]
  {\Large\filcenter\bfseries}{\MakeUppercase{Chapter \thechapter}\\}{0em}{\MakeUppercase{#1}}
  
\titleformat{\section}[block]
  {\large\bfseries}{\thesection}{0.5em}{#1}
  
\titleformat{\subsection}[block]
  {\normalsize\bfseries}{\thesubsection}{0.5em}{#1}
  

\newcommand{\signature}{\rule{3in}{1.2pt}}
\newcommand{\thesistitle}{Affective Motivational Collaboration Theory}
\newdateformat{monthyear}{\monthname[\THEMONTH] \THEYEAR}

\DeclareMathOperator*{\argmin}{\arg\!\min}
\DeclareMathOperator*{\argmax}{\arg\!\max}

\linespread{1.5}

\begin{document}

\begin{titlepage}
\begin{center}
\large\textbf{\thesistitle}\\[0.5em]

\large\textnormal{by}\\
\large\textnormal{Mahni Shayganfar - mshayganfar@wpi.edu}\\[0.5em]

\large\textnormal{A PhD Dissertation}\\[0.5em]

\large\textnormal{Presented at}\\[0.5em]
\large\textsc{WORCESTER POLYTECHNIC INSTITUTE}\\[0.5em]
\large\textnormal{in partial fulfillment of the requirements for the}\\[0.5em]
\large\textnormal{DOCTOR OF PHILOSOPHY}\\[0.5em]
\large\textnormal{in}\\
\large\textnormal{Computer Science}\\
\large\textnormal{November 2016}\\[0.75em]
\end{center}

\noindent\large\textsc{Approved}\\[0.5em]
\Large\textnormal{\signature}\\
\normalsize\textnormal{Professor Charles Rich, Thesis Advisor}\\[0.5em]
\Large\textnormal{\signature}\\
\normalsize\textnormal{Professor Candace L. Sidner, Thesis Co-Advisor}\\[0.5em]
\Large\textnormal{\signature}\\
\normalsize\textnormal{Professor John E. Laird, Thesis Committee
Member}\\[0.5em] \Large\textnormal{\signature}\\
\normalsize\textnormal{Professor Stacy Marsella, Thesis Committee Member}
\end{titlepage}

\thispagestyle{empty}
\vspace*{\fill}
  \begin{center}
    \textcopyright \hspace{0.5em} Copyright by Mahni Shayganfar 2016

    All Rights Reserved
  \end{center}
\vspace*{\fill}
\newpage

\pagenumbering{roman}

\chapter*{Abstract}
\addcontentsline{toc}{chapter}{Abstract}

Abstract Here!

\pagebreak

\chapter*{Acknowledgments}
\addcontentsline{toc}{chapter}{Acknowledgments}

Acknowledgments Here!

\pagebreak

\tableofcontents
\pagebreak

\listoffigures
\pagebreak

\listoftables
\pagebreak

%\listofalgorithms
%\addcontentsline{toc}{chapter}{List of Algorithms}
%\pagebreak

\pagenumbering{arabic}

\chapter{Introduction}
\label{ch:introduction}

\section{Motivation}

\section{Thesis Statement and Scope}

\section{Contributions}

Throughout this work we aim to show how a robot can leverage emotion-driven
processes using appraisal algorithms to improve collaboration with humans. As
such, in this thesis work, we introduce a novel framework, called Affective
Motivational Collaboration (AMC) framework, which allows a robotic agent to
collaborate with humans incooperating the underlying emotion-driven processes
and the expressed emotion of human collaborator. Such a framework is built
based on computational models of collaboration and appraisal allowing for
task-driven interaction with robots or other agents. The theoretical foundation,
computational models and algorithms as well as the overall framework, and the
end-to-end evaluation of the framwork make the following contributions:

\begin{enumerate}
  \item \textbf{Developing new computational models and algorithms for
  \textit{Affective Motivational Collaboration Framework}}:
  
	(Chapters \ref{ch:amct} and \ref{ch:appraisals}) As mentioned earlier, since
	the theoretical foundation of AMC framework is built on the Shared-Plans theory
	of collaboration [cite] and cognitive appraisal theory of emotions [cite], one
	of the contributions of our work is to create computational models and
	algorithms to compute the value of appraisal variables in a dyadic
	collaboration. Applying cognitive appraisal theory in the collaboration context
	is novel. Other models of the appraisal theory have not paid attention to the
	dynamics of the collaboration.

  	We have also developed a new algorithm for the emotion-driven goal management
  	in the context of collaboration. Goal management is one of the important
  	functions of emotions during collaboration. Existing models and
  	implementations of emotions focus only on how emotions regulate and control
  	internal processes and sometimes behaviors. This part of our work shows how
  	appraisal components of the self and the human collaborator contributes to
  	the goal management as an emotion function.
  
  \item \textbf{Developing and implementing a computational model based on
  \textit{Affective Motivational Collaboration Theory}:}

  My computational model will implement the key algorithms related to
  \textit{Affective Motivational Collaboration Theory} as well as minimal
  implementation of other processes which are required for validation of the
  model but are not part of my thesis contributions. The emphasis of the model
  is on underlying cognitive processes embracing collaboration and appraisal
  concepts, rather than the Perception and the Action mechanisms.

  \item \textbf{Validating \textit{Affective Motivational Collaboration
  Theory:}}

  I have identified eight key social characteristics (see Section
  \ref{conceptual-constructs}) which occur during the course of a collaboration.
  I will validate how the various functions of emotions give rise to these
  characteristics during collaboration. Specifically, I will first incrementally
  validate one or more of the computational components in my model starting with
  appraisal. Finally, I will conduct an end-to-end system evaluation with human
  subjects and a simulated robot.
  
\end{enumerate}

\chapter{Background and Related Work}
\label{ch:background}

\section{Computational Collaboration Theories}

\subsection{Shared-Plans Theory}

\subsection{Joint-Intentions Theory}

\subsection{Hybird Theories}

\subsection{Similarities and Differences}

\subsection{Applications of Collaboration Theories}

\section{Affective Computing}

\subsection{Affect and Emotions}

\subsection{Functions of Emotions}

\subsection{Motivation and Theory of Mind}

\section{Computational Models of Emotions}

\subsection{Appraisal Theory}

\subsection{Other Computational Models}

\subsection{Similarities and Differences}

\subsection{Applications in Autonomous Agents and Robots}

\chapter{Affective Motivational Collaboration Theory}
\label{ch:amct}

\section{Introduction}

\subsection{Scenario}

\subsection{Example of a Collaborative Interaction}

\section{Design and Architecture}

\subsection{Mechanisms}

\subsection{Functions of Emotions}

\subsection{Mental States}

\subsection{Attributes of Mental States}

\chapter{Computational Framework}
\label{ch:framework}

\section{System Overview}

\section{Components of the Architecture}

\subsection{Mental States}

\subsection{Collaboration}

\subsection{Appraisal}

\subsection{Coping}

\subsection{Motivation}

\subsection{Theory of Mind}

\subsection{Perception}

\subsection{Action}

\chapter{Appraisal Processes in Collaboration Context}
\label{ch:appraisals}

\section{Introduction}

\section{Appraisal and Collaboration}

\section{Appraisal Algorithms}

\subsection{Relevance}

\subsection{Desirability}

\subsection{Expectedness}

\subsection{Controllability}

\section{Methodology [This chapter will contain the crowdsourding study.]}

\section{Results and Evaluation}

\chapter{Improving Human-Robot Collaboration \\ Using Emotional-Awareness}
\label{ch:awareness}

\section{Introduction}

\section{Collaborative Behaviors and Emotional-Awareness}

\subsection{Goal Postponement}

\subsection{Goal Management}

\subsection{Task Delegation}

\section{Methodology}

\section{Results and Evaluation}

\chapter{Conclusion}
\label{ch:conclusion}

\section{Discussion}

\section{Future Work}

\pagebreak

\bibliographystyle{abbrv}
\bibliography{phd2016-toris}


\begin{appendices}
\chapter*{Appendix A}
\label{apdx:constraints}
\addcontentsline{toc}{chapter}{A}

\end{appendices}

\end{document}