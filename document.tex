\documentclass[12pt]{report}

\usepackage[titletoc]{appendix}
\usepackage{datetime}
\usepackage{graphicx}
\usepackage{amsmath}
\usepackage{algorithmicx}
\usepackage{algorithm}
\usepackage[explicit]{titlesec}
\usepackage{tocloft}
\usepackage{amssymb}
\usepackage{longtable}
\usepackage{multirow}
\usepackage[english]{babel}

\usepackage{subcaption}
\usepackage{subfig}

\usepackage{url}

\renewcommand{\familydefault}{\rmdefault}
\renewcommand\cftchapaftersnum{.}
\renewcommand\cftchapdotsep{\cftdotsep}

\setlength{\textheight}{8.63in}
\setlength{\textwidth}{5.9in}
\setlength{\topmargin}{-0.2in}
\setlength{\oddsidemargin}{0.3in}
\setlength{\evensidemargin}{0.3in}
\setlength{\headsep}{0.0in}

\titleformat{\chapter}[block]
  {\Large\filcenter\bfseries}{\MakeUppercase{Chapter \thechapter}\\}{0em}{\MakeUppercase{#1}}
  
\titleformat{\section}[block]
  {\large\bfseries}{\thesection}{0.5em}{#1}
  
\titleformat{\subsection}[block]
  {\normalsize\bfseries}{\thesubsection}{0.5em}{#1}
  

\newcommand{\signature}{\rule{3in}{1.2pt}}
\newcommand{\thesistitle}{Affective Motivational Collaboration Theory}
\newdateformat{monthyear}{\monthname[\THEMONTH] \THEYEAR}

\DeclareMathOperator*{\argmin}{\arg\!\min}
\DeclareMathOperator*{\argmax}{\arg\!\max}

\linespread{1.5}

\begin{document}

\begin{titlepage}
\begin{center}
\large\textbf{\thesistitle}\\[0.5em]

\large\textnormal{by}\\
\large\textnormal{Mahni Shayganfar - mshayganfar@wpi.edu}\\[0.5em]

\large\textnormal{A PhD Dissertation}\\[0.5em]

\large\textnormal{Presented at}\\[0.5em]
\large\textsc{WORCESTER POLYTECHNIC INSTITUTE}\\[0.5em]
\large\textnormal{in partial fulfillment of the requirements for the}\\[0.5em]
\large\textnormal{DOCTOR OF PHILOSOPHY}\\[0.5em]
\large\textnormal{in}\\
\large\textnormal{Computer Science}\\
\large\textnormal{November 2016}\\[0.75em]
\end{center}

\noindent\large\textsc{Approved}\\[0.5em]
\Large\textnormal{\signature}\\
\normalsize\textnormal{Professor Charles Rich, Thesis Advisor}\\[0.5em]
\Large\textnormal{\signature}\\
\normalsize\textnormal{Professor Candace L. Sidner, Thesis Co-Advisor}\\[0.5em]
\Large\textnormal{\signature}\\
\normalsize\textnormal{Professor John E. Laird, Thesis Committee
Member}\\[0.5em] \Large\textnormal{\signature}\\
\normalsize\textnormal{Professor Stacy Marsella, Thesis Committee Member}
\end{titlepage}

\thispagestyle{empty}
\vspace*{\fill}
  \begin{center}
    \textcopyright \hspace{0.5em} Copyright by Mahni Shayganfar 2016

    All Rights Reserved
  \end{center}
\vspace*{\fill}
\newpage

\pagenumbering{roman}

\chapter*{Abstract}
\addcontentsline{toc}{chapter}{Abstract}

Abstract Here!

\pagebreak

\chapter*{Acknowledgments}
\addcontentsline{toc}{chapter}{Acknowledgments}

Acknowledgments Here!

\pagebreak

\tableofcontents
\pagebreak

\listoffigures
\pagebreak

\listoftables
\pagebreak

%\listofalgorithms
%\addcontentsline{toc}{chapter}{List of Algorithms}
%\pagebreak

\pagenumbering{arabic}

\chapter{Introduction}
\label{ch:introduction}

\section{Motivation}

\section{Thesis Statement and Scope}

In this thesis, we develop and validate a framework based on \textit{Affective
Motivational Collaboration Theory} which can improve the effectiveness of
collaboration between agents/robots and humans. This thesis is established based
on the reciprocal influence of collaboration structure and the appraisal
processes in a dyadic collaboration. We focus only on two-participant
collaboration; teamwork collaboration is out of our scope. Furthermore, this
work focuses on a) the influence of emotion-regulated processes on the
collaboration structure, and b) prediction of the observable behaviors of the
other during a collaborative interaction.

We describe the cognitive processes involved in a collaboration in the context
of a cognitive architecture. There are several well-developed cognitive
architectures, e.g., Soar \cite{laird:soar} and ACT-R \cite{anderson:act-r},
each with different approaches to defining the basic cognitive and perceptual
operations. There have also been efforts to integrate affect into these
architectures \cite{dancy:actR-physiology-affect, marinier:behavior-emotion}. In
general, however, these cognitive architectures do not focus on processes to
specifically produce emotion-regulated goal-driven collaborative behaviors. At
the same time, existing collaboration theories, e.g., SharedPlans
\cite{grosz:plans-discourse} theory, focus on describing the structure of a
collaboration in terms of fundamental mental states, e.g., mutual beliefs or
joint intentions. However, they do not describe the associated processes, their
relationships, and influences on each other. \textit{Affective Motivational
Collaboration Theory} deals with some of the major affect-driven processes
having an impact on the collaboration structure. This theory is informed by
research in psychology and artificial intelligence which is reviewed in Chapter
\ref{ch:background}. Our contribution, generally speaking, is to synthesize
prior work on appraisal and collaboration, and motivation thus to provide a new
theory which describes some of the prominent emotion-regulated goal-driven
phenomena in a dyadic collaboration.

\section{Contributions}

Throughout this work we aim to show how a robot can leverage emotion-driven
processes using appraisal algorithms to improve collaboration with humans. As
such, in this thesis work, we introduce a novel framework, called Affective
Motivational Collaboration (AMC) framework, which allows a robotic agent to
collaborate with humans incooperating the underlying emotion-driven processes
and the expressed emotion of human collaborator. Such a framework is built
based on computational models of collaboration and appraisal allowing for
task-driven interaction with robots or other agents. The theoretical foundation,
computational models and algorithms as well as the overall framework, and the
end-to-end evaluation of the framwork make the following contributions:

\begin{enumerate}
  \item \textbf{Introducing \textit{Affective Motivational Collaboration Theory}:}
  
  	(Chapter \ref{ch:amct}) As mentioned earlier, since the theoretical
  	foundation of AMC framework is built on the combination of SharedPlans
  	theory of collaboration \cite{grosz:plans-discourse} and cognitive appraisal
  	theory of emotions [cite], one of the contributions of our work is to
  	introduce theoretical concepts incorporating key notions of both theories in
  	a dyadic collaboration context. Applying cognitive appraisal theory in the
  	collaboration context is novel. Other models of the appraisal theory have not
  	paid attention to the dynamics of the collaboration.
	
  \item \textbf{Developing new computational models and algorithms for
  \textit{Affective Motivational Collaboration Framework}:}
  
	(Chapter \ref{ch:appraisals}) Another contributions of our work is to create
	computational models and algorithms to compute the value of appraisal variables
	in a dyadic collaboration. We have also developed a new algorithm for the
	emotion-driven goal management in the context of collaboration. Goal management
	is one of the important functions of emotions during collaboration. Existing
	models and implementations of emotions focus only on how emotions regulate and
	control internal processes and sometimes behaviors. This part of our work shows
	how appraisal components of the self and the human collaborator contributes to
  	the goal management as an emotion function.
  
  \item \textbf{Developing a computational framework based on \textit{Affective
  Motivational Collaboration Theory}:}

  (Chapter \ref{ch:framework}) In order to evaluate our computational models and
  algorithms within an interaction with human collaborators, we have developed
  a computational framework based on our theoretical foundations in Affective
  Motivational Collaboration Theory. Our computational framework implements the
  key concepts related to \textit{Affective Motivational Collaboration Theory}
  as well as minimal implementation of other processes which are required for
  validation of the model but are not part of this thesis contributions. The
  emphasis of the model is on underlying cognitive processes embracing
  collaboration and appraisal concepts, rather than the Perception and the
  Action mechanisms.

  \item \textbf{Validating \textit{Affective Motivational Collaboration
  Theory:}}

  (Chapters \ref{ch:appraisals} and \ref{ch:awareness}) We have conducted two
  user studies a) to validate our appraisal algorithms before further
  development of our frmaework, and b) to investigate the overall functionality
  of our framework within an end-to-end system evaluation with human subjects
  and a robot. The second user study was also conducted to evaluate the benefit
  of using our computational framework in human-robot collaboration. In the
  first user study, we crowd sourced our questionnaires to test our hypothesis
  that humans and our algorithms will provide similar answers to questions
  related to different factors within our appraisal algorithms. In the second
  user study, we investigated the importance of emotional awareness in
  human-robot collaboration, and the overall functionality of the AMC framework
  with the participants in our study environment.
  
\end{enumerate}

\chapter{Background and Related Work}
\label{ch:background}

\section{Computational Collaboration Theories}

\subsection{Shared-Plans Theory}

\subsection{Joint-Intentions Theory}

\subsection{Hybird Theories}

\subsection{Similarities and Differences}

\subsection{Applications of Collaboration Theories}

\section{Affective Computing}

\subsection{Affect and Emotions}

\subsection{Functions of Emotions}

\subsection{Motivation and Theory of Mind}

\section{Computational Models of Emotions}

\subsection{Appraisal Theory}

\subsection{Other Computational Models}

\subsection{Similarities and Differences}

\subsection{Applications in Autonomous Agents and Robots}

\chapter{Affective Motivational Collaboration Theory}
\label{ch:amct}

\section{Introduction}

\subsection{Scenario}

\subsection{Example of a Collaborative Interaction}

\section{Design and Architecture}

\subsection{Mechanisms}

\subsection{Functions of Emotions}

\subsection{Mental States}

\subsection{Attributes of Mental States}

\chapter{Appraisal Processes in Collaboration Context}
\label{ch:appraisals}

\section{Introduction}

\section{Appraisal and Collaboration}

\section{Appraisal Algorithms}

\subsection{Relevance}

\subsection{Desirability}

\subsection{Expectedness}

\subsection{Controllability}

\section{Methodology [This chapter will contain the crowdsourding study.]}

\section{Results and Evaluation}

\chapter{Computational Framework}
\label{ch:framework}

\section{System Overview}

\section{Components of the Architecture}

\subsection{Mental States}

\subsection{Collaboration}

\subsection{Appraisal}

\subsection{Coping}

\subsection{Motivation}

\subsection{Theory of Mind}

\subsection{Perception}

\subsection{Action}

\chapter{Improving Human-Robot Collaboration \\ Using Emotional-Awareness}
\label{ch:awareness}

\section{Introduction}

\section{Collaborative Behaviors and Emotional-Awareness}

\subsection{Goal Postponement}

\subsection{Goal Management}

\subsection{Task Delegation}

\section{Methodology}

\section{Results and Evaluation}

\chapter{Conclusion}
\label{ch:conclusion}

\section{Discussion}

\section{Future Work}

\pagebreak

\bibliographystyle{abbrv}
\bibliography{mshayganfar}


\begin{appendices}
\chapter*{Appendix A}
\label{apdx:constraints}
\addcontentsline{toc}{chapter}{A}

\end{appendices}

\end{document}